\documentclass[]{article}
%\usepackage{verbatim}
%\usepackage[dvips]{graphicx}
%\usepackage{pslatex}
\title{Classifying Protein Secondary Structures through Deep Network Learning}
\author{Stephanie deWet and Adam Vail}

\begin{document}
\maketitle

\section{Introduction}
\label{subsec:intro}
Predicting the underlying 3D structure of a protein from its amino acid sequence is a significant problem in structural molecular biology.
An important subproblem of this is to predict complex local structures, or secondary structures, from the amino acid sequence.  
The secondary structure problem has been attacked in a variety of ways, including with x-ray crystallography, biophysical models, and machine learning models.

Within in the field of machine learning, a variety of methods have been applied  over the past twenty years, with varying results. 
The majority of early work was done using standard neural networks with zero or one hidden layers resulting in the vast majority falling below 70\% accuracy.
The recent rise in popularity in deep networks (i.e. neural networks with multiple hidden layers) led us to investigate their accuracy in this problem space.

The common method of encoding amino acids uses large input vectors with an extremely sparse 1-ok-k representation.
Deep networks have the ability to reduce the dimensionality of large sparse feature spaces and are potentially well suited to simplifying the secondary structure problem.

\section{Related Works}
\label{subsec:relatedworks}
One of the earliest studies in this field was conducted by Qian and Sejnowski \cite{Qian} in 1988.
They experimented with using perceptrons and standard back-propagated single hidden layer neural networks.
They achieved results of 63.4\% accuracy which provides us with a reference point for our baselines.

Stolorz et al. combined perceptrons with bayesian methods to probabilistically inject domain knowledge into their model.
The combination of these two methods yielded an improvement over Qian with an accuracy of 64.4\%

Rost and Sander experimented with more complex network structures similar to deep networks.
They created an ensemble of complex networks consisting of two layers of standard single hidden layer neural networks, with the output of the first being the input of the second.
Within the ensemble, each network was trained using varying parameters in order to create separation between the models.
A majority vote was used to produce the final output which was modified in special cases using specific domain knowledge.
This method produced a significant increase in accuracy achieving 70.8\%.

\section{Method}
As alluded to in section \ref{subsec:intro}, our hypothesis is that a deep network has the ability to outperform a standard neural network on the protein secondary structure classification
problem.
Our objective is to find the optimal tuning of the deep network to yield the best predictive accuracy.

\subsection{Deep Networks Consisting of Stacked Autoencoders}
We choose a standard approach to creating a deep network by stacking varying numbers of autoencoders \cite{Hinton} which feed into a set of output units.
An autoencoder is a neural network with a single hidden layer that is trained using backpropagation to reproduce its inputs as outputs.
When there are less hidden units then inputs, the autoencoders enforces a more compact representation of the feature vector.
The process of creating a deep network consists of training a set stacked autoencoders where each successive autoencoder's inputs are the previous autoencoder's hidden layer.

\subsubsection{Input Encoding}
Our training data consists of a set of proteins, each of which is a variable length sequence of amino acids.
Each amino acid in the chain can take on one of twenty discrete values.
Therefore, there are two issues that must be overcome to use an amino acid sequence as input to a neural network: handling variable length inputs and modeling discrete values.

We model the discrete values by using a 1-of-k encoding.
Each amino acid is represented by twenty input nodes, one for each of its potential values.
Of the twenty input nodes, we set the input node corresponding to the amino acid's value to one and set all others to zero.
This was chosen because we did not want to impose an ordering on the values of the amino acid or to make certain values look artificially closer to each other.
Rost and Qian chose a similar encoding.

In order to handle variable length inputs and the fact that protein sequences tend to be quite long, we chose to use a sliding window over the sequence.
For every sliding window the goal is to classify the secondary structure of the amino acid at the center of the window.
The effect of changing the window size changes the impact of neighbors on the central amino acid's classification.
The challenge of using a sliding window is handling the amino acids that reside within half the window size of either end of the sequence.
We have chosen to model the ends of the sequence by setting each nonexistent (due to being off the end of the sequence) amino acid's twenty input nodes all to zero.
This method of using a sliding window was also used by Rost and Qian.

\subsubsection{Output Layer}
Our three classifications are the coarse secondary structures: alpha helix, beta sheet, or loop (known in some of the literature as coil).
We represent each of these possibilities with an output unit, where the highest value of the three output units is chosen in a winner-take-all fashion.
These outputs are attached to the hidden layer of the final autoencoder, and the resultant perceptrons are trained with back propagation.
We use a sigmoid function in all of the hidden and output units.

\subsection{Modifications to the Basic Algorithm}
We have run some tests with each of the following modifications to the deep network.

\subsubsection{Decaying hidden layer size}
Because then input is a large sparse array, a more compact representation is desirable.
We use a decay factor to decrease the size of each stacked hidden layer, incrementally reducing the size of the feature space.
This is designed to find underlying structures in the data.

\subsubsection{Additional Connections to Output Units}
Due to the fact that the autoencoders find a compact representation of the data, we wanted to see whether we were losing valuable information in hidden layers.
We modified the output layer to connect to every input and hidden unit, so that the final layer has inputs from every layer of the deep network.
We then trained the final layer as a giant perceptron, and continued to use a winner-take-all final prediction.

\subsection{Baseline}
Because the related works in section \ref{subsec:relatedworks} all used domain knowledge in some form, we developed our own baseline model for comparison to our deep network.
The baseline is a standar neural network using a single hidden layer and trained with back propagation.

\subsubsection{Ensemble of Neural Networks}
The accuracy gained by Rost from the use of an ensemble inspired us to use an ensemble as well.
Unfortunately, due to the time complexity of convergence and limitations of physical compute resources, we were only able to develop this for the simpler baseline neural network.
A majority vote on the outputs of the individual neural networks was used to determine the final output of the ensemble.

\begin{thebibliography}{9}

\bibitem{Hinton}
G.E. Hinton and R. R. Salakhutdinov,
   `` Reducing the Dimensionality of Data with Neural Networks"
   in \emph{Science}, 2006.

\bibitem{Qian}
Ning Qian and Terrence J. Sejnowski,
  `` Predicting the Secondary Structure of Globular Proteins Using Neural Network Models"
  in \emph{J. Mol. Bio.}, 1988.

\bibitem{Rost}
Burkhard Rost and Christ Sander,
  `` Prediction of Protein Secondary Structures at Bettter than 70\% Accuracy"
  in \emph{J. Mol. Bio.}, 1993.

\bibitem{Stolorz}
Paul Stolorz, Alan Lapedes, and Yuan Xia,
   `` Predicting Protein Secondary Structure Using Neural Net and Statistical Methods"
   in \emph{J. Mol. Bio.}, 1992.

\end{thebibliography}

\end{document}
