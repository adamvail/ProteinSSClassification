\documentclass[]{article}
%\usepackage{verbatim}
%\usepackage[dvips]{graphicx}
%\usepackage{pslatex}
\title{Classifying Protein Secondary Structures through Deep Network Learning}
\author{Stephanie deWet and Adam Vail}

\begin{document}
\maketitle

\section{Goals}
The goal of our project is to classify protein structures through the use of deep networks.
Specifically, we are going to attempt to map each amino acid in the protein to its secondary structure.
Protein classification based upon amino acid sequence is an open problem that is well known in machine learning.
Other learning methods have been applied in the past such as k-nearest neighbor, traditional neural networks (single hidden layer), and support vector machines.
With the recent popularity of deep networks, we hope to discover their level of ability to correctly classify protein structures compared with the more conventional learning algorithms.
If time permits, we hope to use a sequential labeling model such as hidden Markov models or conditional random fields to process the output of the deep network model.

\section{Algorithms}
As mentioned above, we plan to apply deep networks to the problem of protein classification.
We plan on doing this through the use of autoencoders.
Our hope is that the multiple levels of autoencoders will discover comprehensible internal structures that combine to make up the secodary structures.

\section{Experiments}
We plan to format the problem as predicting the secondary structure of an amino acid, given a window of amino acids around it.
We will try solving the problem using both a traditional neural net with back propagation and a deep network with autoencoders.
Our goal is to compare the two methods, and see whether deep networks provide a benefit over the traditional neural networks.

\section{Datasets}
We plan to use the 130 non-redundant proteins selected in \cite{Rost}, because they have already been selected intelligently.
We will use the CATH database to get the amino acid sequence for proteins, because it has a clean and convenient API to get this sequence, and
the Database of Secondary Structure in Proteins (DSSP) to provide the correct secondary structure for our training data. 

\begin{thebibliography}{9}

\bibitem{Rost}
Burkhard Rost and Christ Sander,
  `` Prediction of Protein Secondary Structures at Bettter than 70\% Accuracy"
  in \emph{J. Mol. Bio.}, 1993.

\end{thebibliography}

\end{document}
